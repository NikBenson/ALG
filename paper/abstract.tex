\begin{abstract}
    Mit dem Wachstums des Quat\"aren Wirtschaftssektor, auch Datensektor genannt, werden Daten und derren Haushalt immer wichtiger. Für die meisten dieser Unternehmen sind heutzutage Big Data oder das Data Warehouse wichtige Themen.\\
    Aber desto mehr Daten verarbeitet werden, desto wichtiger ist es auch, dass effiziente Algorithmen verwendet werden. Vergleichsbasierte Sortieralgorithmen haben eine theoretische optimale Laufzeit von $O(n\log{n})$. Dem kommen Algorithmen wie der Ouicksort mit einer eben solchen durchschnittlichen Laufzeit auch sehr nahe, im schlechtesten Falle ist es beim Beispiel des Quicksorts allerdings $O(n^2)$.\\
    Hier schafft Samplesort \autocite{frazer-1970} abhilfe, indem die Daten randomisiert werden.
    Dies ist gerade bei vorsortierten Datens\"atzen ein wichtiger Vorteil.\\
    Im Folgenden ist dargestellt, wie Samplesort in verschiedenen Varrianten funktioniert und wann es sinnvoll ist, diesen zu nutzen.
\end{abstract}
