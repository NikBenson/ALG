\begin{abstract}
    Mit dem Wachstum des Quatären Wirtschaftssektor, auch Datensektor genannt, werden Daten und deren Haushalt immer wichtiger.
    Für die meisten dieser Unternehmen sind heutzutage Big Data oder das Data Warehouse wichtige Themen.\\
    Aber desto mehr Daten verarbeitet werden, desto wichtiger ist es auch, dass effiziente Algorithmen verwendet werden.
    Vergleichsbasierte Sortieralgorithmen haben eine theoretische optimale Laufzeit von $O(\log_2{n!})$. Dem kommen Algorithmen wie Ouicksort mit einer durchschnittlichen Laufzeit von $O(n\log{n})$ auch sehr nahe, im schlechtesten Falle ist es beim Beispiel Quicksort allerdings $O(n^2)$.\\
    Hier schafft Samplesort \autocite{frazer-1970} Abhilfe, indem die Daten randomisiert werden.
    Dies ist gerade bei vorsortierten Datensätzen oder wenn der Sortierprozess parallelisiert ablaufen soll ein wichtiger Vorteil.\\
    Im Folgenden ist dargestellt, wie Samplesort in verschiedenen Varianten funktioniert und wann es sinnvoll ist, diesen zu nutzen.
\end{abstract}
