\section{Mathematische Eigenschaften}
	Wie die meisten Sortieralgorithmen ist Samplesort mathematisch darstellbar.
	Hierbei gibt es viele Perspektiven.
	Im Folgenden ist die Laufzeit- und Speicherkomplexität dargestellt.
	
	\subsection{Speicherkomplexität}
		Samplesort ist ein in-place Sortieralgorithmus.
		Das heißt, es wird kein zusätzlicher Speicher benötigt.\\
		Soweit die Theorie zur Standardvariante.
		Anders ist es beim Super Scalar Sample Sort. \autocite{sanders-2004} 
		Dieser benötigt einen 2. temporären Array mit n Elementen und damit doppelt so viel Speicher.
		An der Speicherkomplexität ändert sich damit allerdings nichts, diese bleibt bei $\Theta(n)$.
	\subsection{Laufzeitkomplexität}

		
