\section{Einleitung}
    Samplesort ist ein Vergleichsbasierter Sortieralgorithmus, der 1970 zuerst vorgestellt wurde. \autocite{frazer-1970}
    %TODO
	\subsection{Sortierproblematik}
	Schon seit anbeginn der Theoretischen Informatik sind Sortieralgoritmen weit diskutiert.
	Heute sind viele verschiedene Algorithmen bekannt, um verschiedene Anreihungen von Daten, auf die unterschiedlichsten Arten zu sortieren.
	Eine Eigenschaft hat sich dabei besonders herausgestellt, und so wird im Folgenden nur Bezug auf vergleichsbasierte Sortieralgorithmen genommen.
	Diese m\"ussen ein Minimum von $n\log{n}$ Vergleichen durchführen, um jedes Element eindeutig einordnen zu k\"onnen, und haben damit eine minimale Laufzeitkomplexit\"at von $\Omega(n\log{n})$.
	\footnote{$O(f(x))$ ist die maximal begrenzte, $\Omega(f(x))$ die minimal begrenzte und $\Theta(f(x))$ die übereinstimmende maximal und minimal begrenzte Laufzeit. $O$, $\Omega$ und $\Theta$ beschreiben dabei einen konstanten Faktor. \autocite[4]{sedgewick-1996}}
	
	\paragraph{Aspekte eines Sortieralgorithmus}
	Die auswahl des richtigen Algoritmus kann dabei, abhängig vom Verwendungsfall, von verschiedenen Faktoren abhängig sein.\\
	Die klassischen Beispiele sind \textbf{Laufzeit} und benötigter zusätzlicher \textbf{Speicher}.
	Aber auch ob der Algoritmus \textbf{stabil} ist, also die uhrsprüngliche Reihenfolge gleichwertiger Elemente gewährleitet ist, können eine Rolle spielen.

    \subsection{Was ist Samplesort?}
    Bei all diesen Kriterien schneidet Quicksort im Schnitt am besten ab, stößt aber auch schnell an seine Grenzen. Sind die Daten in umgekehrter sortierter Reihenvolge vorhanden, so steigert sich seine Laufzeit von durchschnittlich $O(n\log{n})$ auf $\Omega(n^2)$.\\
    Dieser Fall erscheint zunächst selten, doch wenn zum Beispiel neue Nutzerdaten jeden Tag um Mitternacht neu in eine strukturierte Datenbank übernommen werden sollen, ist dies häufig der Fall.\\
    Heapsort kann hier, mit einer Laufzeit von $\Theta(n\log{n})$ Abhilfe schaffen, ist aber im Schnitt langsammer als Quicksort.\\
    Samplesort baut deshalb auf Quicksort auf:\\
    W\"arend Quicksort ein Pivot Element hat, hat Samplesort $p$ Pivot Elemente, auch Splitter genannt, wobei diese zuf\"allig ausgew\"alt werden.
    Dadurch ben\"otigt Samplesort im Schnitt 15\% weniger Vergleiche als Quicksort. \autocite{frazer-1970}
    Außerdem ist Samplesort f\"ur Systeme mit mehreren Prozessorkernen ausgelegt.
    So wird $p$ meistens als die Anzahl von Kernen - 1 deffiniert und durch die zuf\"allige Auswahl der Splitter hat jeder Kern eine ann\"aernd identische Last und der Algoritmus kann den vollen Prozessor nutzen, um schneller fertig zu werden.
    
