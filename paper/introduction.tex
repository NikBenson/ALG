\section{Einleitung}
	\subsection{Sortierproblematik}
	Schon seit anbeginn der Theoretischen Informatik sind Sortieralgoritmen weit diskutiert.
	Heute sind viele verschiedene Algorithmen bekannt, um verschiedene Anreihungen von Daten, auf die unterschiedlichsten Arten zu sortieren.
	Eine Eigenschaft hat sich dabei besonders herausgestellt, und so ist im Folgenden nur bezug auf vergleichsbasierte Sortieralgorithmen genommen.
	Diese m\"ussen ein Minimum von $n\log{n}$ Vergleichen durchführen, um jedes Element eindeutig einordnen zu k\"onnen, und haben damit eine minimale Laufzeitkomplexit\"at von $\Omega(n\log{n})$.
	\footnote{$O(f(x))$ ist die maximal begrenzte, $\Omega(f(x))$ die minimal begrenzte und $\Theta(f(x))$ die übereinstimmende maximal und minimal begrenzte Laufzeit. $O$, $\Omega$ und $\Theta$ beschreiben dabei einen konstanten Faktor.}
	
	\paragraph{Bestandteile eines Sortieralgorithmus}
	Bestandteile Eines Algorithmus...

    \subsection{Was ist Samplesort?}
	
    \subsection{Vorteil gegenüber anderen Sortieralgorithmen}
