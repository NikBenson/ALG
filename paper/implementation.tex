\section{Implementierung}
Es gibt viele verschiedene Varriationen, Samplesort zu implementieren. Zu viele, um sie hier alle zu nennen.
Im Folgenden sind die bekanntesten Varriationen dargestellt.\\
Dazu sind zun\"achst ein paar Grundbegriffe von n\"oten:
\paragraph{Splitter}
Splitter wurden bereits in der Einleitung erwähnt.
Dort haben wir uns damit begn\"ugt, diese mit den Pivot Elementen beim Quicksort zu vergleichen.
Die Konzepte sind dabei sehr \"ahnlich.\\
W\"arend Quicksort die Daten mit dem Pivot Element in 2 Sektionen unterteilt, da bin\"ar die einfachste aufteilungsm\"oglichkeit ist, wenn die Anzahl irrelevant ist, unterteilt Samplesort die Daten mit $n$ Splittern in $n+1$ Buckets.
\paragraph{Bucket}
Ein Bucket ist der bereich zwischen zwei benachbarten Splittern, so wie die beiden bereiche neben dem gr\"oßten sowie kleinsten Splitter.\\
Wenn $\{10, 20, 30\}$ die 3 Splitter sind, dann ergeben sich daraus also die Buckets $\{(-\infty;10],\allowbreak [10;20],\allowbreak [20;30],\allowbreak [30;\infty)\}$.\\
Wichtig ist dabei, dass Elemente, die den selben Wert, wie ein Splitter haben, entweder in den Bucket links, oder rechts, von jenem, eingeordnet werden k\"onnen.\\
Das Ziel ist dabei h\"aufig, die Anzahl der Buckets identisch zur Anzahl der verf\"ugbaren Prozessorkerne zu setzen, damit die Buckets parallel sortiert werden. Dies ist in \ref{sec:multithreading} weiter ausgef\"uhrt.

\subsection{Grundger\"ust}
Der grundlegende Aufbau ist dabei immer identisch und in \ref{fig:skeletton} zu erkennen.
\lstinputlisting[language=C, caption=Samplesort Implementierung, label=fig:skeletton]{../code/samplesort.c}

\subsection{Auswahl der Splitter}

\subsubsection{Oversampeling}

\lstinputlisting[language=C, caption=Auswahlverfahren der Splitter, label=fig:select_splitters_bloc]{../code/select_splitters.c}

\subsection{Zuordnen jeden Wertes zu dem passenden Bucket}
\lstinputlisting[language=C, caption=Finden des richtigen Buckets]{../code/find_bucket_index_for_value_by_binary_search.c}

\subsection{Sortieren der Buckets}
\subsubsection{Original}
\subsubsection{Rekursiv}
